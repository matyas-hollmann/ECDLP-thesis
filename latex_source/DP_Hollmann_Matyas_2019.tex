% arara: xelatex
% arara: xelatex
% arara: xelatex


% options:
% thesis=B bachelor's thesis
% thesis=M master's thesis
% czech thesis in Czech language
% english thesis in English language
% hidelinks remove colour boxes around hyperlinks

\documentclass[thesis=M,english]{FITthesis}[2012/10/20]

\usepackage[utf8]{inputenc} % LaTeX source encoded as UTF-8
% \usepackage[latin2]{inputenc} % LaTeX source encoded as ISO-8859-2
% \usepackage[cp1250]{inputenc} % LaTeX source encoded as Windows-1250

\usepackage{graphicx} %graphics files inclusion
% \usepackage{subfig} %subfigures
\usepackage{amsmath} %advanced maths
\usepackage{amssymb} %additional math symbols

\usepackage{dirtree} %directory tree visualisation

\usepackage{amsthm} 
\theoremstyle{remark}
\newtheorem*{RM}{Remark}

\newtheorem*{NRM}{Notational Remark}

\theoremstyle{definition}
\newtheorem{DF}{Definition}[section]

% % list of acronyms
% \usepackage[acronym,nonumberlist,toc,numberedsection=autolabel]{glossaries}
% \iflanguage{czech}{\renewcommand*{\acronymname}{Seznam pou{\v z}it{\' y}ch zkratek}}{}
% \makeglossaries

% % % % % % % % % % % % % % % % % % % % % % % % % % % % % % 
% EDIT THIS
% % % % % % % % % % % % % % % % % % % % % % % % % % % % % % 

\department{Department of Information Security}
\title{Summation polynomials and the discrete logarithm problem on elliptic curve}
\authorGN{Matyáš} %author's given name/names
\authorFN{Hollmann} %author's surname
\author{Matyáš Hollmann} %author's name without academic degrees
\authorWithDegrees{Bc. Matyáš Hollmann} %author's name with academic degrees
\supervisor{Ing. Ivo Petr, Ph.D.}
\acknowledgements{THANKS (remove entirely in case you do not with to thank anyone)}
\abstractEN{Summarize the contents and contribution of your work in a few sentences in English language.}
\abstractCS{V n{\v e}kolika v{\v e}t{\' a}ch shr{\v n}te obsah a p{\v r}{\' i}nos t{\' e}to pr{\' a}ce v {\v c}esk{\' e}m jazyce.}
\placeForDeclarationOfAuthenticity{Prague}
\keywordsCS{Replace with comma-separated list of keywords in Czech.}
\keywordsEN{Replace with comma-separated list of keywords in English.}
\declarationOfAuthenticityOption{4} %select as appropriate, according to the desired license (integer 1-6)
% \website{http://site.example/thesis} %optional thesis URL


\begin{document}

% \newacronym{CVUT}{{\v C}VUT}{{\v C}esk{\' e} vysok{\' e} u{\v c}en{\' i} technick{\' e} v Praze}
% \newacronym{FIT}{FIT}{Fakulta informa{\v c}n{\' i}ch technologi{\' i}}

\setsecnumdepth{part}
\chapter{Introduction}

\setsecnumdepth{all}
\chapter{Mathematical background}\label{mathBG}
%\input{mathBG.tex}
 %definice pojmu, finite fields, etc.
 In this chapter we are going to define terms that will be used in the rest of this thesis. The first part is focused on terms common in general algebra, the second part will deal with elliptic curves and a little bit of algebraic geometry. 
\section{Introduction to general algebra}
%\newtheorem{corollary}{Corollary}[theorem] dusledek
%\newtheorem{lemma}[theorem]{Lemma} poznamka pouzita k dukazu
\begin{DF}
A \textbf{group} $G$ is an ordered pair $(M,  \circ)$, where $M$ is a non-empty set and binary operation $\circ : M \times M \to M $ (sometimes called the group law of $G$) that satisfies three requirements known as group axioms: 
\end{DF}
\begin{itemize}
\item 
$ \forall x,y,z \in M: x\circ (y \circ z) = (x \circ y) \circ z,$ \hfill (associativity)
\item 
$ \exists e \in M, \forall x \in M: e \circ x = x \circ e = x,$ \hfill (identity element)
\item 
$\forall x \in M, \exists x^{-1} \in M: x \circ x^{-1} = x^{-1} \circ x = e.$ \hfill (inverse element)
\end{itemize}
\begin{RM}
$M$ is closed under the operation $\circ$. 
\end{RM}
\begin{NRM}
When we are gonna talk about an element  $g$ of a group $G$ ($g \in G$) we are actually gonna mean that $g$ is an element of the underlying set $M$ ($g \in M$).
\end{NRM}
Groups satisfying commutativity law:
\begin{itemize}
\item 
$ \forall x, y\in M: x \circ y = y \circ x,$
\end{itemize}
are called \textbf{Abelian groups} (in honour of a famous Norwegian mathematician Niels Henrik Abel \cite{Abel}). 
\begin{DF}
If the set $M$ has a finite number of elements, $G = (M, \circ)$ is a \textbf{finite} group. \textbf{Order} of the finite group $G$ is the number of elements of the underlying set $M$ and we denote it by $\#G$. If the set $M$ is infinite, the order of $G$ is infinite as well.
\end{DF}
\begin{RM}
In every group there exist just one unique identity element. Also for every element $q \in G$ there exists just one inverse element denoted $q^{-1}$ in the multiplicative notation and $-q$ in the additive notation. Inverse of a product of two group elements is a product of respective inverses in the reversed order (order does matter in non-commutative groups).  
\end{RM}
\noindent Identity element in additive notation is called \textbf{zero} and denoted by $0$, in the multiplicative notation \textbf{unit} and denoted $1$. \\ \\
In an additive group $G$ we define \textbf{multiplication} by an integer (repeated application of the group law) as follows:
$$
\forall p \in G,\ \forall k \in \mathbb{Z}: kp := \begin{cases} \underbrace{p + p + \cdots + p}_{\text{k-times}} &\quad k > 0, \\
0 \text{ (identity element) } &\quad k = 0, \\
\underbrace{(-p) + (-p) + \cdots + (-p)}_{\text{k-times}} &\quad k < 0.
\end{cases}
$$
In a multiplicative group $G$ we define \textbf{exponentiation} (repeated application of the group law) in a similar manner:
$$
\forall p \in G,\ \forall k \in \mathbb{Z}: p^k := \begin{cases} \underbrace{p \cdot p \cdot \cdots \cdot p}_{\text{k-times}} &\quad k > 0, \\
1 \text{ (identity element) } &\quad k = 0, \\
\underbrace{p^{-1} \cdot p^{-1} \cdot \cdots \cdot p^{-1}}_{\text{k-times}} &\quad k < 0.
\end{cases}
$$
\begin{DF}
\textbf{Order of an element} $a \in G$ is the smallest positive integer $k$ such that: $a^k = 1$ (similarly $ka = 0$ in additive notation), we denote it by $\#a= k$, if there isn't such $k$ we say the order of $a$ is infinite (this case may only happen if $G$ itself is of infinite order and we are mostly interested in finite groups in this thesis). Elements of finite order are sometimes called \textbf{torsion} elements.
\end{DF}
\begin{RM}
Order of the identity element $\in G$ is always $1$ and due to the uniqueness of the identity element it's also the only element $\in G$ this order.
\end{RM}
\chapter{Discrete logarithm problem on elliptic curves}
\chapter{State-of-the-art}

\chapter{Algorithms}
Pollard-Rho
Pohling-Hellmann
BabySteps-Giants (mention mods)
F4, F5 Groebner
SumPoly \cite{iso690}
\chapter{Analysis and design}

\chapter{Realisation}

\setsecnumdepth{part}
\chapter{Conclusion}


%\bibliographystyle{iso690}
%\bibliography{bibliography}
\begin{thebibliography}{10}
\bibitem{secTest}{GUPTA, Kamlesh and Sanjay SILAKARI. ECC over RSA for Asymmetric Encryption: A Review. International Journal of Computer Science Issues. 2011, 8(05). Also available at: \url{https://pdfs.semanticscholar.org/45ef/c85c2a92fe00dae5477f918795d9fd23e6e5.pdf}.}

\bibitem{algGeom} {COX, David A., John LITTLE and Donal O'SHEA. \textit{Ideals, Varieties, and Algorithms: An Introduction to Computational Algebraic Geometry and Commutative Algebra.} Fourth Edition. New York: Springer, 2015. Undergraduate texts in mathematics. ISBN 978-3-319-16721-3.}

\bibitem{mky} {KALVODA, Tomáš, Ivo PETR and Štěpán STAROSTA. Matematika pro kryptologii [online]. KAM FIT ČVUT. [Praha], Updated on 20-02-2019 [Accessed on 16-04-2019]. Available at: \url{https://courses.fit.cvut.cz/MI-MKY/media/lectures/mi-mky-poznamky-v17.pdf}}

\bibitem{myBP} {HOLLMANN, Matyáš. \textit{Implementace násobení na neasociativních (nekomutativních) algebrách.} Praha, 2017. Bakalářská práce. České vysoké učení technické v Praze, Fakulta informačních technologií. Vedoucí práce Jiřina Scholtzová. Also available at: \url{https://dspace.cvut.cz/bitstream/handle/10467/69263/F8-BP-2017-Hollmann-Matyas-thesis.pdf}}

\bibitem{bruno}{BUCHBERGER, Bruno. Bruno Buchberger's PhD thesis 1965: An Algorithm for Finding the Basis Elements of the Residue Class Ring of a Zero Dimensional Polynomial Ideal. Journal of Symbolic Computation. 2006, 41(03), 475-511. DOI: \url{https://doi.org/10.1016/j.jsc.2005.09.007}}

\bibitem{week11}{BROWN, Gavin. \textit{Zero-dimensional ideals}. Chapter 9 in Polynomials in Several Variables - MA574 (course notes) [online]. 2016. [Accessed on 16-04-2019]. Available at: \url{https://www.kent.ac.uk/smsas/personal/gdb/MA574/week11.pdf}.}

\bibitem{handbook}{COHEN, Henri, Gerhard FREY and Roberto AVANZI. \textit{Handbook of elliptic and hyperelliptic curve cryptography.} Boca Raton: Taylor \& Francis, 2006. ISBN 978-1-58488-518-4.}

\bibitem{shoup}{SHOUP, Victor. \textit{Lower Bounds for Discrete Logarithms and Related Problems.} Advances in Cryptology — EUROCRYPT ’97. EUROCRYPT 1997. Springer, Berlin, Heidelberg, 1997. DOI: \url{https://doi.org/10.1007/3-540-69053-0_18}.}

\bibitem{polProb}{BOS, Joppe W., Alina DUDEANU and Dimitar JETCHEV. \textit{Collision bounds for the additive Pollard rho algorithm for solving discrete logarithms.} Journal of Mathematical Cryptology. 2014, 8(1), 71-92. ISSN (Online) 1862-2984. DOI: \url{https://doi.org/10.1515/jmc-2012-0032}.}

\bibitem{PH}{POHLIG, S. and M. HELLMAN. \textit{An improved algorithm for computing logarithms over $GF(p)$ and its cryptographic significance (Corresp.)}. IEEE Transactions on Information Theory. 1978, 24(1), 106-110.  ISSN (Online) 1557-9654 DOI: \url{https://doi.org/10.1109/TIT.1978.1055817}.}

\bibitem{subExp}{GALBRAITH, Steven D. \textit{Mathematics of public key cryptography.} New York: Cambridge University Press, 2012. ISBN 978-1-107-01392-6.}

\bibitem{amadori17}{AMADORI, Alessandro, Federico PINTORE and Massimiliano SALA. \textit{On the discrete logarithm problem for prime-field elliptic curves}. Finite fields and their applications. 2018, 51(May), 168-182. DOI: \url{https://doi.org/10.1016/j.ffa.2018.01.009}.}

\bibitem{semaev04}{SEMAEV, Igor. \textit{Summation polynomials and the discrete logarithm problem on elliptic curves.} Cryptology ePrint Archive, Report 2004/031, 2004. Also available at: \url{https://eprint.iacr.org/2004/031}.}

\bibitem{semaev15}{SEMAEV, Igor. \textit{New algorithm for the discrete logarithm problem on elliptic curves.} Cryptology ePrint Archive, Report 2015/310, 2015. Also available at \url{https://eprint.iacr.org/2015/310}.}

\bibitem{FGLM}{FAUGÉRE, J.C., P. GIANNI, D. LAZARD and T. MORA. \textit{Efficient Computation of Zero-dimensional Gröbner Bases by Change of Ordering}. Journal of Symbolic Computation. 1994, 11(1-000). Also available at: \url{https://www-polsys.lip6.fr/~jcf/Papers/FGLM.pdf}.}

\bibitem{grobComp}{CAMINATA, Alessio and Elisa GORLA. \textit{Solving Multivariate Polynomial Systems and an Invariant from Commutative Algebra.} Cryptology ePrint Archive, Report 2017/593, 2017. Also available at: \url{https://eprint.iacr.org/2017/593}.}

\bibitem{macal}{JÓNSSON, Gudbjörn F. and Stephen A. VAVASIS. \textit{Accurate solution of polynomial equations using Macaulay resultant matrices}. Mathematics of Computation. 2004, 74(249), 221-262. Also available at:
\url{http://www.ams.org/journals/mcom/2005-74-249/S0025-5718-04-01722-3/S0025-5718-04-01722-3.pdf}.}

\bibitem{guire}{McGUIRE, Gary and Daniela MUELLER. \textit{A New  Index Calculus Algorithm for the Elliptic Curve Discrete Logarithm Problem and Summation Polynomial Evaluation.} Cryptology ePrint Archive, Report 2017/1262, 2017. Also available at: \url{https://eprint.iacr.org/2017/1262}.}

\bibitem{magma}{BOSMA, Wieb, John CANNON and Catherine PLAYOUST, \textit{Magma Computational Algebra System.} [online] Copyright © 2010-2019. Accessed on [01-05-2019]. Available at: \url{http://magma.maths.usyd.edu.au/magma/}.}

\bibitem{gbComp}{PEARCE, Roman. \textit{Roman Pearce: Groebner Bases}. [online] 2016. Accessed on [01-05-2019].
Available at: \url{http://www.cecm.sfu.ca/~rpearcea/mgb.html}.}

\bibitem{sage}{STEIN, W. A. et al., \textit{SageMath Mathematical Software System - Sage (Version 8.7 (2019-03-23)).}
The Sage Development Team, 2019 [online]. Accessed on [01-05-2019]. Available at: \url{http://www.sagemath.org/}.}

\bibitem{brick}{BRICKENSTEIN, Michael. \textit{Slimgb: Gröbner bases with slim polynomials.} Revista Matemática Complutense. 2010, 23(2), 453-466. ISSN 1139-1138. DOI: \url{http://link.springer.com/10.1007/s13163-009-0020-0}.}

\bibitem{fgb}{FAUG\`{E}RE, J.-C., \textit{FGb: A Library for Computing Gröbner Bases. Version 1.68 (build 14539). [online]} Updated on 31.07.2015. Accessed on [10-04-2019]. Available at: \url{https://www-polsys.lip6.fr/~jcf/FGb/index.html}.}

\bibitem{giac}{PARISSE, Bernard and Renée De GRAEVE, \textit{Giac/Xcas (Version 1.5.0-3)} [online]. Updated in September 2018. Accessed on [15-04-2019]. Available at: \url{https://www-fourier.ujf-grenoble.fr/~parisse/giac.html}.}

\end{thebibliography}
%citace

\setsecnumdepth{all}
\appendix

\chapter{Acronyms}
% \printglossaries
\begin{description}
	\item[GUI] Graphical user interface
	\item[XML] Extensible markup language
\end{description}


\chapter{Contents of enclosed CD}

%change appropriately

\begin{figure}
	\dirtree{%
		.1 readme.txt\DTcomment{the file with CD contents description}.
		.1 exe\DTcomment{the directory with executables}.
		.1 src\DTcomment{the directory of source codes}.
		.2 wbdcm\DTcomment{implementation sources}.
		.2 thesis\DTcomment{the directory of \LaTeX{} source codes of the thesis}.
		.1 text\DTcomment{the thesis text directory}.
		.2 thesis.pdf\DTcomment{the thesis text in PDF format}.
		.2 thesis.ps\DTcomment{the thesis text in PS format}.
	}
\end{figure}

\end{document}
