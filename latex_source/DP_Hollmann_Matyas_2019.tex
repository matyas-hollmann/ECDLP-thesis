% arara: xelatex
% arara: xelatex
% arara: xelatex


% options:
% thesis=B bachelor's thesis
% thesis=M master's thesis
% czech thesis in Czech language
% english thesis in English language
% hidelinks remove colour boxes around hyperlinks

\documentclass[thesis=M,english]{FITthesis}[2012/10/20]

\usepackage[utf8]{inputenc} % LaTeX source encoded as UTF-8
% \usepackage[latin2]{inputenc} % LaTeX source encoded as ISO-8859-2
% \usepackage[cp1250]{inputenc} % LaTeX source encoded as Windows-1250

\usepackage{graphicx} %graphics files inclusion
% \usepackage{subfig} %subfigures
\usepackage{amsmath} %advanced maths
\usepackage{amssymb} %additional math symbols

\usepackage{dirtree} %directory tree visualisation

\usepackage{amsthm} 
\theoremstyle{remark}
\newtheorem*{RM}{Remark}

\newtheorem*{NRM}{Notational Remark}

\theoremstyle{definition}
\newtheorem{DF}{Definition}[section]

% % list of acronyms
% \usepackage[acronym,nonumberlist,toc,numberedsection=autolabel]{glossaries}
% \iflanguage{czech}{\renewcommand*{\acronymname}{Seznam pou{\v z}it{\' y}ch zkratek}}{}
% \makeglossaries

% % % % % % % % % % % % % % % % % % % % % % % % % % % % % % 
% EDIT THIS
% % % % % % % % % % % % % % % % % % % % % % % % % % % % % % 

\department{Department of Information Security}
\title{Summation polynomials and the discrete logarithm problem on elliptic curve}
\authorGN{Matyáš} %author's given name/names
\authorFN{Hollmann} %author's surname
\author{Matyáš Hollmann} %author's name without academic degrees
\authorWithDegrees{Bc. Matyáš Hollmann} %author's name with academic degrees
\supervisor{Ing. Ivo Petr, Ph.D.}
\acknowledgements{THANKS (remove entirely in case you do not with to thank anyone)}
\abstractEN{Summarize the contents and contribution of your work in a few sentences in English language.}
\abstractCS{V n{\v e}kolika v{\v e}t{\' a}ch shr{\v n}te obsah a p{\v r}{\' i}nos t{\' e}to pr{\' a}ce v {\v c}esk{\' e}m jazyce.}
\placeForDeclarationOfAuthenticity{Prague}
\keywordsCS{Replace with comma-separated list of keywords in Czech.}
\keywordsEN{Replace with comma-separated list of keywords in English.}
\declarationOfAuthenticityOption{4} %select as appropriate, according to the desired license (integer 1-6)
% \website{http://site.example/thesis} %optional thesis URL


\begin{document}

% \newacronym{CVUT}{{\v C}VUT}{{\v C}esk{\' e} vysok{\' e} u{\v c}en{\' i} technick{\' e} v Praze}
% \newacronym{FIT}{FIT}{Fakulta informa{\v c}n{\' i}ch technologi{\' i}}

\setsecnumdepth{part}
\chapter{Introduction}

\setsecnumdepth{all}
\chapter{Mathematical background}\label{mathBG}
%\input{mathBG.tex}
 %definice pojmu, finite fields, etc.
 V~této kapitole definujeme základní matematické pojmy použité v~této práce. Čerpali jsme především z~přednášek \cite{BI-LIN} a skript \cite{LIN-skripta} k~předmětu BI-LIN, handoutů k~předmětu MI-MPI \cite{MI-MPI}, disertační práce naší vedoucí \cite{disertace} a článků \cite{nil3}, \cite{niln}. Ostatní zdroje jsou uvedeny explicitně v~místě, kde byly použity.

\section{Introduction to general algebra}
%\newtheorem{corollary}{Corollary}[theorem] dusledek
%\newtheorem{lemma}[theorem]{Lemma} poznamka pouzita k dukazu
\begin{DF}
A \textbf{group} $G$ is an ordered pair $(M,  \circ)$, where $M$ is any non-empty set and binary operation $\circ : M \times M \to M $ (sometimes called the group law of $G$) that satisfies three requirements known as group axioms: 
\end{DF}
\begin{itemize}
\item 
$ \forall x,y,z \in M: x\circ (y \circ z) = (x \circ y) \circ z,$ \hfill (associativity)
\item 
$ \exists e \in M, \forall x \in M: e \circ x = x \circ e = x,$ \hfill (identity element)
\item 
$\forall x \in M, \exists x^{-1} \in M: x \circ x^{-1} = x^{-1} \circ x = e.$ \hfill (inverse element)
\end{itemize}
\begin{RM}
$M$ is closed under an operation $\circ$. 
\end{RM}
\begin{NRM}
When we are gonna talk about an element  $g$ of a group $G$ ($g \in G$) we are actually gonna mean that $g$ is an element of the underlying set $M$ ($g \in M$).
\end{NRM}
Groups satisfying commutativity law:
\begin{itemize}
\item 
$ \forall x, y\in M: x \circ y = y \circ x,$
\end{itemize}
are called \textbf{Abelian groups} (in honour of a famous Norwegian mathematician Niels Henrik Abel) \url{https://www.britannica.com/biography/Niels-Henrik-Abel}.
\\
\\


\chapter{Discrete logarithm problem on elliptic curves}
\chapter{State-of-the-art}

\chapter{Analysis and design}

\chapter{Realisation}

\setsecnumdepth{part}
\chapter{Conclusion}


\bibliographystyle{iso690}
\bibliography{mybibliographyfile}

\setsecnumdepth{all}
\appendix

\chapter{Acronyms}
% \printglossaries
\begin{description}
	\item[GUI] Graphical user interface
	\item[XML] Extensible markup language
\end{description}


\chapter{Contents of enclosed CD}

%change appropriately

\begin{figure}
	\dirtree{%
		.1 readme.txt\DTcomment{the file with CD contents description}.
		.1 exe\DTcomment{the directory with executables}.
		.1 src\DTcomment{the directory of source codes}.
		.2 wbdcm\DTcomment{implementation sources}.
		.2 thesis\DTcomment{the directory of \LaTeX{} source codes of the thesis}.
		.1 text\DTcomment{the thesis text directory}.
		.2 thesis.pdf\DTcomment{the thesis text in PDF format}.
		.2 thesis.ps\DTcomment{the thesis text in PS format}.
	}
\end{figure}

\end{document}
